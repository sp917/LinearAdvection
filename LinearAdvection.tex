\documentclass[10pt]{article}

\input{Preamble}


\begin{document}

\title{Numerical Methods Assignment 1\\Stuart Patching}

\section{The Linear Advection Equation}

We aim to solve the equation:
\eq{
	u_{t} + au_{x} = 0
	}
with initial condition:
\eq{
	u(x,0) = f(x)
	}
It is straightforward to show that this has the exact solution:
\eq{
	u(x,t) = f(x - ct)
}
We shall, however, use this equation to examine and compare some different numerical methods.

\section{Numerical Schemes}

In order to solve the linear advection equation numerically, we must choose a scheme in which we replace derivatives by discrete iterations. For simplicity we restrict to the domain:
	\eq{
		(x,t) \in [0,1]\times[0,1]
	}
We use the following notation:
\eq{
	u(n\Delta x, m\Delta t) \approx u^{m}_{n}
}

\subsection{CNCS}

From the numerical schemes for the advection equation given in \cite{MPE} on page 292 we see that that there are two methods with second order accuracy in both space and time, and one of these is CNCS which is given by:
\eqlab{
	u^{m+1}_{n} = u^{m}_{n} - \frac{c}{4}\lrb{u^{m+1}_{n+1} - u^{m+1}_{n-1} + u^{m}_{n+1} - u^{m}_{n-1}}  \qquad m = 0,...,M-1, \quad n=1,...,N-1
}{CNCS}
Where:
\eq{
	c = a\frac{\Delta t}{\Delta x}
}
We also need to impose the initial condition, which in this case is:
\eq{
	u^{0}_{n} = f(n\Delta x)
}
We also impose periodic boundary conditions so that $u^{m}_{-1} = u^{m}_{N}$. The effect of this is to make \eqref{eq:CNCS} valid for $n=0$ also. In the exact solution this amounts to extending $f$ periodically. \\

We can now reformulate \eqref{eq:CNCS} as a matrix equation:
\eq{
	A u ^{m+1} = B u^{m}
}
where:
\algn{
	A_{nk} & = \delta_{nk} +  \tfrac{c}{4}(\delta_{(n+1)k} - \delta_{(n-1)k}) \\
	B_{nk} & = \delta_{nk} -  \tfrac{c}{4}(\delta_{(n+1)k} - \delta_{(n-1)k}) 
}
So:
\eq{
	u^{m+1} = A^{-1}Bu^{m}
}



\iffalse
\subsection{BTBS}

The BTBS scheme is reported in \cite{MPE} as having first-order accuracy in both time and space and like CNCS is an implicit method. It is given by:
\eq{
	u^{m+1}_{n} = u^{m}_{n} - c\lrb{u^{m+1}_{n} - u^{m+1}_{n-1}}
}
Using the same boundary and initial conditions as above, we re-write this as:
\eq{
	u^{m+1} = C^{-1}u^{m}
}
Where:
\eq{
	C_{nk}  =  \delta_{nk} + c(\delta_{nk} - \delta_{(n-1)k})
}

\subsection{CTCS}

We consider also the second-order accurate explicit CTCS method:
\eq{
	u^{m+1}_{n} = u^{m-1}_{n} - c(u^{m}_{n+1} - u^{m}_{n-1})
}
This is a two-step method so we use FTCS for the first iteration:
\eq{
	u^{1}_{n} = u^{0}_{n} - \tfrac{c}{2}(u^{0}_{n+1} - u^{0}_{n-1})
}
Thus we have overall:
\algn{
	u^{1} & = Du^{0}\\
	u^{m+1} & = u^{m-1} + Eu^{m} \qquad \text{for}\; m = 1,...,M-1
}
Where:
\algn{
	D_{nk} & = \delta_{nk} - \tfrac{c}{2}(\delta_{(n+1)k} - \delta_{(n-1)k}) \\
	E_{nk} & = -c(\delta_{(n+1)k} - \delta_{(n-1)k})
}

\fi

\subsection{FTBS}

Finally we consider the FTBS method, which is explicit and first order in both time and space. It is given by:
\eq{
	u^{m+1}_{k} = u^{m}_{n} - c(u^{m}_{n} - u^{m}_{n-1})
}
i.e.:
\eq{
	u^{m+1} = Fu^{m}
}
\eq{
	F_{nk} = (1-c)\delta_{nk} + c\delta_{(n-1)k}
}

\subsection{Results}

Here we print the $t=1$ graph for the two methods described above for $a = 0.7$, $\Delta x = \Delta t = 0.001$ and with initial condition: 

\eq{f(x) = \exp\lrb{-100(x-0.1)^{2}}}

\begin{figure}[H]
\centering
\includegraphics[width=0.5\textwidth]{solution}
\caption{Write caption}
\end{figure}

We can see from this that CNCS is much closer to the exact solution than FTBS. This is what we would expect, as it is a higher-order method. However, consider different boundary conditions, such as:
\eq{
	f(x) = H((x-0.1)(0.2-x))
}
where $H(x) = 1$ if $x>0$ and zero otherwise. The results of this are then:

\begin{figure}[H]
\centering
\includegraphics[width=0.5\textwidth]{solutiondiscont}
\caption{Write caption}
\end{figure}

We see that in the case in which the initial condition contains discontinuities, the FTBS method produces a solution closer to the exact solution. 

\subsection{Order of Accuracy}
\subsubsection{CNCS}
\begin{table}[H]
	\centering
		\begin{tabular}{|c|c|c|}
			\hline
							$\Delta t$        & \text{Error}		&	\text{Gradient}\\
			\hline
                                0.000100        & 0.000442743 & n/a \\ 
			\hline
                                0.000250        & 0.000448399 & 0.013853732 \\ 
			\hline
                                0.000500        & 0.000468597 & 0.063564234 \\ 
			\hline
                                0.000750        & 0.000502127 & 0.170443554 \\ 
			\hline
                                0.001000        & 0.000549351 & 0.312446659 \\ 
			\hline
                                0.002500        & 0.001112865 & 0.770449170 \\ 
			\hline
                                0.005000        & 0.003094616 & 1.475481727 \\ 
			\hline
                                0.007500        & 0.006302162 & 1.754106072 \\ 
			\hline
                                0.010000        & 0.010669559 & 1.830152179 \\ 
			\hline
                                0.025000        & 0.043986127 & 1.545884436 \\ 
			\hline
                                0.050000        & 0.083318826 & 0.921593926 \\ 
			\hline
                                0.075000        & 0.102721159 & 0.516304840 \\ 
			\hline
                                0.100000        & 0.115640894 & 0.411814057 \\ 
              \hline
		\end{tabular}
		\caption{put something sensible here}
\end{table}

Graphically we find the following:

\begin{figure}[H]
\centering
\includegraphics[width=0.5\textwidth]{logerror}
\caption{Write caption}
\end{figure}

\subsubsection{FTBS}



\subsection{Changing the Courant Number}

Now let us examine what happens when we use, for example $c = 1.1$, with the continuous initial condition used previously.

\begin{figure}[H]
\centering
\includegraphics[width=0.5\textwidth]{c11}
\caption{Write caption}
\end{figure}

So here it seems that the FTBS method is unstable for this value of $c$.

And for negative values such as $c = -0.1$: 

\begin{figure}[H]
\centering
\includegraphics[width=0.5\textwidth]{cneg01}
\caption{Write caption}
\end{figure}

This is consistent with \cite{MPE}, which states that the FTBS method is stable only for $c\in[0,1]$, whereas CNCS is stable for all $c\in \R$.

\begin{thebibliography}{9}
	\bibitem{MPE} D. Crisan et al. 2017. \i{Mathematics of Planet Earth: A Primer}
\end{thebibliography}

\end{document}